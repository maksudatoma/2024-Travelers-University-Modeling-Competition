% Options for packages loaded elsewhere
\PassOptionsToPackage{unicode}{hyperref}
\PassOptionsToPackage{hyphens}{url}
\PassOptionsToPackage{dvipsnames,svgnames,x11names}{xcolor}
%
\documentclass[
  10pt,
  letterpaper,
  twocolumn]{article}

\usepackage{amsmath,amssymb}
\usepackage{iftex}
\ifPDFTeX
  \usepackage[T1]{fontenc}
  \usepackage[utf8]{inputenc}
  \usepackage{textcomp} % provide euro and other symbols
\else % if luatex or xetex
  \usepackage{unicode-math}
  \defaultfontfeatures{Scale=MatchLowercase}
  \defaultfontfeatures[\rmfamily]{Ligatures=TeX,Scale=1}
\fi
\usepackage{lmodern}
\ifPDFTeX\else  
    % xetex/luatex font selection
  \setmainfont[]{Times New Roman}
\fi
% Use upquote if available, for straight quotes in verbatim environments
\IfFileExists{upquote.sty}{\usepackage{upquote}}{}
\IfFileExists{microtype.sty}{% use microtype if available
  \usepackage[]{microtype}
  \UseMicrotypeSet[protrusion]{basicmath} % disable protrusion for tt fonts
}{}
\makeatletter
\@ifundefined{KOMAClassName}{% if non-KOMA class
  \IfFileExists{parskip.sty}{%
    \usepackage{parskip}
  }{% else
    \setlength{\parindent}{0pt}
    \setlength{\parskip}{6pt plus 2pt minus 1pt}}
}{% if KOMA class
  \KOMAoptions{parskip=half}}
\makeatother
\usepackage{xcolor}
\setlength{\emergencystretch}{3em} % prevent overfull lines
\setcounter{secnumdepth}{-\maxdimen} % remove section numbering
% Make \paragraph and \subparagraph free-standing
\ifx\paragraph\undefined\else
  \let\oldparagraph\paragraph
  \renewcommand{\paragraph}[1]{\oldparagraph{#1}\mbox{}}
\fi
\ifx\subparagraph\undefined\else
  \let\oldsubparagraph\subparagraph
  \renewcommand{\subparagraph}[1]{\oldsubparagraph{#1}\mbox{}}
\fi


\providecommand{\tightlist}{%
  \setlength{\itemsep}{0pt}\setlength{\parskip}{0pt}}\usepackage{longtable,booktabs,array}
\usepackage{calc} % for calculating minipage widths
% Correct order of tables after \paragraph or \subparagraph
\usepackage{etoolbox}
\makeatletter
\patchcmd\longtable{\par}{\if@noskipsec\mbox{}\fi\par}{}{}
\makeatother
% Allow footnotes in longtable head/foot
\IfFileExists{footnotehyper.sty}{\usepackage{footnotehyper}}{\usepackage{footnote}}
\makesavenoteenv{longtable}
\usepackage{graphicx}
\makeatletter
\def\maxwidth{\ifdim\Gin@nat@width>\linewidth\linewidth\else\Gin@nat@width\fi}
\def\maxheight{\ifdim\Gin@nat@height>\textheight\textheight\else\Gin@nat@height\fi}
\makeatother
% Scale images if necessary, so that they will not overflow the page
% margins by default, and it is still possible to overwrite the defaults
% using explicit options in \includegraphics[width, height, ...]{}
\setkeys{Gin}{width=\maxwidth,height=\maxheight,keepaspectratio}
% Set default figure placement to htbp
\makeatletter
\def\fps@figure{htbp}
\makeatother

\usepackage{sdss2020} % Uses Times Roman font (either newtx or times package)
\usepackage{url}
\usepackage{hyperref}
\usepackage{latexsym}
\usepackage{amsmath, amsthm, amsfonts}
\usepackage{algorithm, algorithmic}
\usepackage[dvipsnames]{xcolor} % colors
\newcommand{\mt}[1]{{\textcolor{blue}{#1}}}
\newcommand{\svp}[1]{{\textcolor{RedOrange}{#1}}}
\makeatletter
\makeatother
\makeatletter
\makeatother
\makeatletter
\@ifpackageloaded{caption}{}{\usepackage{caption}}
\AtBeginDocument{%
\ifdefined\contentsname
  \renewcommand*\contentsname{Table of contents}
\else
  \newcommand\contentsname{Table of contents}
\fi
\ifdefined\listfigurename
  \renewcommand*\listfigurename{List of Figures}
\else
  \newcommand\listfigurename{List of Figures}
\fi
\ifdefined\listtablename
  \renewcommand*\listtablename{List of Tables}
\else
  \newcommand\listtablename{List of Tables}
\fi
\ifdefined\figurename
  \renewcommand*\figurename{Figure}
\else
  \newcommand\figurename{Figure}
\fi
\ifdefined\tablename
  \renewcommand*\tablename{Table}
\else
  \newcommand\tablename{Table}
\fi
}
\@ifpackageloaded{float}{}{\usepackage{float}}
\floatstyle{ruled}
\@ifundefined{c@chapter}{\newfloat{codelisting}{h}{lop}}{\newfloat{codelisting}{h}{lop}[chapter]}
\floatname{codelisting}{Listing}
\newcommand*\listoflistings{\listof{codelisting}{List of Listings}}
\makeatother
\makeatletter
\@ifpackageloaded{caption}{}{\usepackage{caption}}
\@ifpackageloaded{subcaption}{}{\usepackage{subcaption}}
\makeatother
\makeatletter
\@ifpackageloaded{tcolorbox}{}{\usepackage[skins,breakable]{tcolorbox}}
\makeatother
\makeatletter
\@ifundefined{shadecolor}{\definecolor{shadecolor}{rgb}{.97, .97, .97}}
\makeatother
\makeatletter
\makeatother
\makeatletter
\makeatother
\ifLuaTeX
  \usepackage{selnolig}  % disable illegal ligatures
\fi
\IfFileExists{bookmark.sty}{\usepackage{bookmark}}{\usepackage{hyperref}}
\IfFileExists{xurl.sty}{\usepackage{xurl}}{} % add URL line breaks if available
\urlstyle{same} % disable monospaced font for URLs
\hypersetup{
  pdftitle={Final Report},
  pdfauthor={Maksuda Aktar Toma; Aarif Baksh},
  colorlinks=true,
  linkcolor={blue},
  filecolor={Maroon},
  citecolor={Blue},
  urlcolor={Blue},
  pdfcreator={LaTeX via pandoc}}

\title{Final Report}
\author{
Maksuda Aktar Toma\\
Statistics Department, University of Nebraska, Lincoln\\
{\tt \href{mailto:mtoma2@huskers.unl.edu}{mtoma2@huskers.unl.edu}}\\
\\\And
Aarif Baksh\\
Statistics Department, University of Nebraska, Lincoln\\
{\tt \href{mailto:abaksh2@unl.edu}{abaksh2@unl.edu}}\\
}
\date{}

\begin{document}
\maketitle
\ifdefined\Shaded\renewenvironment{Shaded}{\begin{tcolorbox}[enhanced, borderline west={3pt}{0pt}{shadecolor}, frame hidden, interior hidden, sharp corners, breakable, boxrule=0pt]}{\end{tcolorbox}}\fi

\hypertarget{business-problem}{%
\subsection{Business Problem}\label{business-problem}}

As an employee of CloverShield Insurance company, you are tasked with
addressing the challenge of reducing call center costs. Your business
partners have requested the development of a predictive model that,
based on the provided segmentation, forecasts the number of times a
policyholder is likely to call. This model aims to optimize resource
allocation and enhance cost-efficiency in call center operations.

To find all our works on this project go to this link
\url{https://github.com/maksudatoma/2024-Travelers-University-Modeling-Competition/tree/main}

\hypertarget{introduction}{%
\subsection{Introduction}\label{introduction}}

The data obtained from Kaggle, is split into two parts: training data
and validation data. In the validation data, the target variable,
call\_counts, is omitted. The training dataset contains 80,000 samples,
and the validation dataset contains 20,000 samples.

\textbf{Variable Descriptions}

\begin{itemize}
\item
  \texttt{ann\_prm\_amt}: Annualized Premium Amount
\item
  \texttt{bi\_limit\_group}: Body injury limit group (SP stands for
  single split limit coverage, CSL stands for combined single limit
  coverage)
\item
  \texttt{channel}: Distribution channel
\item
  \texttt{newest\_veh\_age}: The age of the newest vehicle insured on a
  policy (-20 represents non-auto or missing values)
\item
  \texttt{geo\_group}: Indicates if the policyholder lives in a rural,
  urban, or suburban area
\item
  \texttt{has\_prior\_carrier}: Did the policyholder come from another
  carrier
\item
  \texttt{home\_lot\_sq\_footage}: Square footage of the policyholder's
  home lot
\item
  \texttt{household\_group}: The types of policy in household
\item
  \texttt{household\_policy\_counts}: Number of policies in the
  household
\item
  \texttt{telematics\_ind}: Telematic indicator (0 represents auto
  missing values or didn't enroll and -2 represents non-auto)
\item
  \texttt{digital\_contacts\_ind}: An indicator to denote if the policy
  holder has opted into digital communication
\item
  \texttt{12m\_call\_history}: Past one year call count
\item
  \texttt{tenure\_at\_snapshot}:Policy active length in month
\item
  \texttt{pay\_type\_code}: Code indicating the payment method
\item
  \texttt{acq\_method}:The acquisition method (Miss represents missing
  values)
\item
  \texttt{trm\_len\_mo}: Term length month
\item
  \texttt{pol\_edeliv\_ind}: An indicator for email delivery of
  documents (-2 represents missing values)
\item
  \texttt{aproduct\_sbtyp\_grp}: Product subtype group
\item
  \texttt{product\_sbtyp}: Product subtype
\item
  \texttt{call\_counts}: The number of call count generated by each
  policy (target variable)
\end{itemize}

\hypertarget{data-cleaning-and-missing-value-count}{%
\subsection{Data Cleaning and Missing Value
count}\label{data-cleaning-and-missing-value-count}}

First, we prepares the data by cleaning and transforming it (e.g.,
converting characters to factors, marking missing values.)

\begin{longtable}[]{@{}ll@{}}
\caption{\textbf{Table 1: Variables with Missing Values}}\tabularnewline
\toprule\noalign{}
Variable & Number of missing values \\
\midrule\noalign{}
\endfirsthead
\toprule\noalign{}
Variable & Number of missing values \\
\midrule\noalign{}
\endhead
\bottomrule\noalign{}
\endlastfoot
acq\_method & 16,066 \\
newest\_veh\_age & 58,015 \\
pol\_edeliv\_ind & 838 \\
telematics\_ind & 58,015 \\
\end{longtable}

\textbf{Zero Values:} 50.18\% of the rows in the call\_counts column are
zeros, indicating that most customers made no calls. This is significant
and might suggest using models like Zero-Inflated Poisson (ZIP) to
handle the high frequency of zeros.

\textbf{Key Takeaways} - The dataset contains both numeric and
categorical variables, with some columns having significant missing
values. - The target variable (call\_counts) is heavily zero-inflated
and skewed, which may require specialized modeling approaches. - Some
numeric variables, like ann\_prm\_amt and home\_lot\_sq\_footage, have
wide ranges and outliers, suggesting that data transformation or scaling
may be beneficial.


\bibliographystyle{sdss2020} % Please do not change the bibliography style

\end{document}
